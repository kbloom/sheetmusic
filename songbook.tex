\documentclass[letterpaper]{memoir}
\usepackage{pdfpages}
\usepackage{polyglossia}
\setsecnumdepth{none}

\usepackage{fontspec}
\setmainfont{Frank Ruehl CLM}
\setmonofont{Miriam Mono CLM}
\setsansfont{Simple CLM}

\setdefaultlanguage{english}
\setotherlanguage{hebrew}

\title{\RL{ותקעתם בחצוצרות} \\
Utkatem BaChatzotzrot \\
The Pizmonim Fakebook \\
For B$\flat$ Trumpet
}
\author{Chanoch Bloom}

% Takes 2 arguments
%  #1: The display name for the table of contents
%  #2: The PDF file to include.
\newcommand{\song}[2]{\includepdf[pages=-,pagecommand={\thispagestyle{plain}},addtotoc={1,section,1,#1,#2}]{#2}}
\usepackage{enumitem}
\setlist[itemize,1]{label={\fontfamily{cmr}\fontencoding{T1}\selectfont\textbullet}}
\begin{document}

\begin{titlingpage}
\maketitle
\end{titlingpage}

\setlength{\footskip}{72pt}
\pagestyle{plain}
\tableofcontents*

\chapter{Forward}

Rav Ovadia Yosef writes in Yechave Da'at 5:2 that the major principle of 
Sephardic music is found in the verse 
"\RL{על ערבים בתוכה תלינו כינורתינו}" (Tehillim 137:2), expressing that the 
Sephardic musical tradition is based on Arabic songs and Arabic scales (maqamat).
Sephardic Jewish music is a rich tradition that includes a great variety of 
songs with original lyrics for a great variety of moods and occasions, written 
by rabbis and chazzanim through a long stretch of Jewish history.

In addition to 
singing these songs at Shabbat meals and in synagogue, I have taken an interest 
to playing these songs on my trumpet. In trying to learn these songs for my 
trumpet, found it necessary to have sheet music to help me learn and practice.
However, because the melodies for these songs are largey an oral tradition,
I discovered that there is a great lack of sheet music. The few books that are 
available either focus on giving a view of Sephardic songs that come from many 
different Sephardic countries, or they focus largely on Ladino songs, and don't 
delve deeply into the repertoire of songs that are familiar in Middle Eastern 
Sephardic synagogues as I would like. I therefore made it my business to learn 
to transcribe these songs while learning to play them.

The lyrics to these songs are more widely available than the sheet music. A
very partial list of books where the lyrics can be found includes:

\begin{itemize}
   \item Abiaa Renanot
   \item Shir u'Shbacha, Hallel v'Zimrah -- The Syrian ``Red Book'', also online 
    (with recordings) at www.pizmonim.org.
   \item Shir u'Shbacha by R' Moshe Batzri.
\end{itemize}


We sing many of these songs on Shabbat and chagim at these shuls, others 
we know because we apply them to the tefillah on Shabbat. Others I've 
learned by looking on YouTube or pizmonim.org to find something that's 
in my song book and fills a particular thematic need. In particular, 
I've tried to learn a variety of songs for each holiday to sing at meals 
with my family, above and beyond what is usually sung in these synagogues.

Some of the maqamat used in this book are generally tuned to the 12-note chromatic
scale used in Western and European music, such as Ajam, Nahawand, Kurd, and 
Hijaz. (Some of these scales are even used in Western music.)
Other maqamat, such as Rast, Bayat, Sigah, and Saba call for
the musician to play quarter tones. While I have not avoided songs 
written in these scales, I have focused on transcribing them in a way that 
sounds reasonably good on Western instruments, specifically on my trumpet, and 
therefore I have not notated quarter tones in my music, and instead chosen a 
note on the Western chromatic scale that works well enough in practice as a 
substitute. Listening to recordings of these songs, and adjusting the intonation 
by lipping notes up and down, or through use of the valve slides, may help some 
songs to sound better.

My sheet music does not included chords for harmonic accompaniment, because 
harmonic accompaniment is largely absent from the Arabic music tradition based
on maqamat. The maqamat that require quarter tones are generally hostile to 
pleasing harmony, and even in maqamat that aren't hostile to harmony, the 
hallmark of this style of music is to please the listener through melodic 
devices, rather than harmonic devices. (For those interested in learning more about
Arabic music theory, I recommend reading \textit{Inside Arabic Music} by Johnny 
Faraj and Sami Abu Shumays, and their companion website at www.maqamworld.com.)

Velvel Pasternak is a giant in Jewish sheet music. The songs that he has 
transcribed span the gamut from Chassidic to Sephardic to modern 
Israeli. His books aim to cover a breadth of different Sephardic 
traditions and include many popular Middle Eastern Sephardic songs, but 
they don't delve as deeply into Middle Eastern music as I have here.
I have a couple of his Sephardic books and have tried to avoid 
duplicating any song that he has already included in the songbooks I 
own.
As a result, there are many songs in his Sephardic songbooks 
that we do sing in Middle Eastern Sephardic synagogues, including some 
of the more popular songs, so the reader is advised to look at his 
songbooks in addition to my own.

I don't consider myself to be an expert in Arabic music, and there are 
certainly others in the shuls I frequent that have a broader knowledge of this 
genre than I do, but I hope that others will find these transcriptions helpful.

I have chosen the name \RL{ותקעתם בחצוצרות} to allude to the verse
\RL{וביום שמחתכם ובמועדיכם ובראשי חדשיכם ותקעתם בחצצרת על עלתיכם ועל זבחי שלמיכם והיו לכם לזכרון לפני אלקיכם אני ה' אלקיכם}
(Bemidbar 10:10), to refer to the songs for Shabbat, holidays, and other simchas 
that are in the book, and to allude to the trumpet that I play them on.

--Chanoch Bloom


\chapter{Shabbat}
\song{Ashir La'el Asher Shavat}{Ashir_Lael_Asher_Shavat.pdf}

\chapter{Turkish Melodies for Shabbat}
\song{Yoducha Rayonai}{Yoducha_Rayonai_Turkish_with_octave_up.pdf}
\song{Yom HaShabbat Ein Kamohu}{Yom_HaShabbat_Ein_Kamohu_Turkish.pdf}
\song{Ki Eshmerah Shabbat}{Ki_Eshmerah_Shabbat_Turkish.pdf}

\chapter{Bakashot for Shabbat}
\song{Yom Zeh Shiru La'el}{Yom_Zeh_Shiru_Lael.pdf}
\song{Yoducha Rayonai}{Yoducha_Rayonai.pdf}
\song{Yoducha Rayonai \#2}{Yoducha_Rayonai_2.pdf}
\song{Yedid Nefesh}{Yedid_Nefesh.pdf}
\song{Agadelcha}{Agadelcha.pdf}

\chapter{Songs for any simcha}
\song{Eshal Elohai}{Eshal_Elohai.pdf}
\song{Malki Tzuri El Kabir}{Malki_Tzuri_El_Kabir.pdf}
\song{Yahaloma}{Yahaloma.pdf}
\song{Refa Tziri}{Refa_Tziri.pdf}
\song{Lach Ana Orech}{Lach_Ana_Orech.pdf}
\song{Ana b'Chasdecha}{Ana_bChasdecha.pdf}
\song{Eleicha Ekra Yah}{Eleicha_Ekra_Yah.pdf}
\song{N'imah Li}{Nimah_Li.pdf}
\song{Im Ninalu}{Im_Ninalu.pdf}
\song{Nagila Halleluyah}{Nagila_Halleluyah.pdf}
\song{Ani Likrat}{Ani_Likrat.pdf}
\song{Semach Beni B'Chelkecha}{Semach_Beni_BChelkecha.pdf}
\song{Goel Yavo}{Goel_Yavo.pdf}
\song{Ata El Kabir}{Ata_El_Kabir.pdf}
\song{Havivi Yah Havivi}{Havivi_Yah_Havivi.pdf}
\song{Hayyim U'Madon}{Hayyim_UMadon.pdf}
\song{Yerushalyim Ir HaBirah}{Yerushalyim_Ir_HaBirah.pdf}
\song{Boee B'rinah}{Boee_Brinah.pdf}
\song{Yinon Shemo}{Yinon_Shemo.pdf}
\song{El Galil}{El_Galil.pdf}
\song{Yotzer MiYado}{Yotzer_MiYado_2.pdf}
\song{Ozrenei El Chai}{Ozrenei_El_Chai.pdf}
\song{Im HaGolan}{Im_HaGolan.pdf}
\song{Riva Riva}{Riva_Riva.pdf}
\song{Kol Et Eleicha}{Kol_Et_Eleicha.pdf}
\song{Ochil Yom Yom Yerushalyim}{Ochil_Yom_Yom_Yerushalyim.pdf}
\song{Ani Alayich Ayuma}{Ani_Alayich_Ayuma.pdf}
\song{Eli Yah Eli}{Eli_Yah_Eli.pdf} % possibly omit, because it's unclear whether I did it in the right key
\song{L'David Shir U'Mizmor}{LDavid_Shir_uMizmor.pdf}


\chapter{Shalosh Regalim}
\song{Mauzi}{Mauzi.pdf}
\song{Samachti Tehillim 122}{Samachti.pdf}

\chapter{Pesach}
\song{Yachid Norah}{Yachid_Norah.pdf}
\song{Rachum Atah}{Rachum_Atah.pdf}
\song{B'neh Li Zevul Mishkani}{Bneh_Li_Zevul_Mishkani.pdf}
\song{Emunim Irchu Shevach (Yerushalmi)}{Emunim_Irchu_Shevach_Yerushalyim.pdf}
\song{Emunim Irchu Shevach (Syrian)}{Emunim_Irchu_Shevach_Syrian.pdf}
\song{Mi Yimalel Gevurotecha}{Mi_Yimalel_Gevurotecha.pdf}

\chapter{Lag LaOmer}
\song{V'amartem Ko Lachai}{Vamartem_Ko_Lachai.pdf}

\chapter{Shavuot}
\song{Roe' Ne'eman Hu}{Roe_Neeman_Hu.pdf}
\song{T'nu Kavod LaTorah}{Tnu_Kavod_LaTorah.pdf}

\chapter{Sukkot}
\song{Sukkah v'Lulav (Moroccan)}{Sukkah_vLulav_Moroccan.pdf}
\song{Sukkah v'Lulav (Yerushalmi)}{Sukkah_vLulav_Yerushalmi.pdf}
\song{Chanun Rachem}{Chanun_Rachem.pdf}
\song{Yah Et Sukkat David Takim}{Yah_Et_Sukkat_David_Takim.pdf}

\chapter{Chanukah}
\song{Yah Hatzel Yonah}{Yah_Hatzel_Yonah.pdf}
\song{Heichalo Heichalo}{Heichalo_Heichalo.pdf}
\song{L'neri}{Lneri.pdf}

\chapter{Purim}
\song{Ronu Gilu}{Ronu_Gilu.pdf}
\song{Or Gilah}{Or_Gilah.pdf} % might not get included because it's in maqam sigah
\song{El Melech Ne'eman}{El_Melech_Neeman.pdf}
\song{Ezer Mitzarai}{Ezer_Mitzarai.pdf}
\song{Simeni Rosh}{Simeni_Rosh.pdf}
\song{Chish Misgabi Geulah}{Chish_Misgabi_Geulah.pdf}
\song{Eli Tzur Yishuati}{Eli_Tzur_Yishuati.pdf}

\chapter{Weddings}
\song{Yismach Hatani}{Yismach_Hatani.pdf}
\song{El Me'od Na'alah}{El_Meod_Naalah.pdf}
\song{Im Chacham Libecha Beni}{Im_Chacham_Libecha_Beni.pdf}
\song{Makhelot Am}{Makhelot_Am.pdf}
\song{Ya'alah Ya'alah (Yerushalmi)}{Yaalah_Yaalah_Yerushalmi.pdf}
\song{Ya'alah Ya'alah (Syrian)}{Yaalah_Yaalah_Syrian.pdf}
\song{Et Dodim Kalah}{Et_Dodim_Kalah.pdf}

\chapter{Brit Milah}
\song{Mah Tov Mah Na'im}{Mah_Tov_Mah_Naim.pdf}
\song{Yehi Shalom b'Cheleinu (Yerushalmi)}{Yehi_Shalom_bCheleinu_Yerushalmi.pdf}
\song{Yehi Shalom B'Cheleinu (Syrian)}{Yehi_Shalom_bCheleinu_Syrian.pdf}

\chapter{Zeved HaBat}
\song{Nava Yafa Tz'viyah}{Nava_Yafa_Tzviyah.pdf}

\chapter{From Tefillah}
\song{Azharot}{Azharot.pdf}
\song{Halleluyah Tehillim 150}{Halleluyah_Tehillim_150.pdf}
\song{Pitchu Li}{Pitchu_Li.pdf}

\end{document}
