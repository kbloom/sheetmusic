\documentclass[letterpaper]{memoir}
\usepackage{pdfpages}
\usepackage{polyglossia}
\usepackage{hyperref}
\usepackage{ifoddpage}
\setsecnumdepth{none}

\usepackage{fontspec}
\setmainfont{Frank Ruehl CLM}
\setmonofont{Miriam Mono CLM}
\setsansfont{Simple CLM}

% Work around broken \nameref interaction between memoir and hyperref
% from https://tex.stackexchange.com/a/648577
\makeatletter
\RenewDocumentCommand\nameref{s}
    {\IfBooleanTF{#1}{\@namerefstar}{\T@nameref}}
\makeatother

\setdefaultlanguage{english}
\setotherlanguage{hebrew}

\title{\RL{ותקעתם בחצוצרות} \\
Utkatem BaChatzotzrot \\
The Pizmonim Fakebook \\
For B$\flat$ Trumpet
}
\author{Chanoch Bloom}

% Takes 2 arguments
%  #1: The display name for the table of contents
%  #2: The PDF file to include. This PDF name is used as a label for cross references.
\newcommand{\song}[2]{\includepdf[pages=-,offset=3mm 0,pagecommand={\thispagestyle{plain}},addtotoc={1,section,1,#1,#2}]{#2}}

\newcommand{\songref}[1]{\nameref{#1}, page \pageref{#1}}

% The first argument is a 2-page song that is forced onto facing pages.
% The second argument is a 1-page song whose location may be swapped to 
% make that possible.
% The first page of a two-page song needs to be on an even page. 
% Apparently, the page starts changes when we start writing the song, so
% if we're currently on an odd page, then the first song will start on
% an even page.
\newcommand{\forcefacingpages}[2]{
   \checkoddpage
   \ifoddpage
   #1
   #2
   \else
   #1
   #2
   \fi
}


\usepackage{enumitem}
\setlist[itemize,1]{label={\fontfamily{cmr}\fontencoding{T1}\selectfont\textbullet}}
\begin{document}

\begin{titlingpage}
\maketitle
\end{titlingpage}

\setlength{\footskip}{72pt}
\pagestyle{plain}
\tableofcontents*

\chapter{Forward}

\begin{RTL}
\begin{quotation}
\begin{center}
וביום שמחתכם ובמועדיכם ובראשי חדשיכם
\textbf{ותקעתם בחצצרת} 
על עלתיכם ועל זבחי שלמיכם והיו לכם לזכרון לפני אלקיכם אני ה' אלקיכם
\end{center}
\end{quotation}
\end{RTL}

When I first encountered Sephardic music, I was taken by the breadth of Jewish 
music that I had never heard before, and that is mostly unknown outside of the 
Sephardic world. I fell in love with the music and started trying to learn as 
much as I could. Having such a wide repretoire of music available has enriched 
my Shabbat table, and my Yom Tov, as well as the synagogue experience.

When I took an interest to playing these songs on my trumpet, I found it 
necessary to have sheet music to help me learn and practice. This is when I 
discovered that melodies for Sephardic songs, like Arabic music in general, are 
mostly an oral tradition. Though there are great collections of recordings available,
very little of it has been written down as sheet music.
The few existing collections of sheet music for 
Sephardic songs, mostly don't intersect with the music that's popular in the 
communities I am a part of. The books that are available either focus on giving 
a view of Sephardic songs that come from many different Sephardic countries, or 
they focus largely on Ladino songs, and don't delve deeply into the repertoire 
of songs that are familiar in Middle Eastern Sephardic synagogues as I would 
like. I therefore made it my business to learn to transcribe these songs while 
learning to play them.

Rav Ovadia Yosef writes in Yechave Da'at 5:2 that the major principle of 
Sephardic music is found in the verse 
"\RL{על ערבים בתוכה תלינו כינורתינו}" (Tehillim 137:2), expressing that the 
Sephardic musical tradition is based on Arabic songs and Arabic scales 
(maqamat).
Some of the maqamat used in this book are generally tuned to the 12-note chromatic
scale used in Western and European music, such as Ajam, Nahawand, Kurd, and 
Hijaz. (Some of these scales are even used in Western music.)
Other maqamat, such as Rast, Bayat, Sigah, and Saba call for
the musician to play quarter tones. While I have not avoided songs 
written in these scales, I have focused on transcribing them in a way that 
sounds reasonably good on Western instruments, specifically on my trumpet, and 
therefore I have not notated quarter tones in my music, and instead chosen a 
note on the Western chromatic scale that works well enough in practice as a 
substitute. Listening to recordings of these songs, and adjusting the intonation 
by lipping notes up and down, or through use of the valve slides, may help some 
songs to sound better.

My sheet music does not included chords for harmonic accompaniment, because 
harmonic accompaniment is largely absent from the Arabic music tradition based
on maqamat. The maqamat that require quarter tones are generally hostile to 
pleasing harmony, and even in maqamat that aren't hostile to harmony, the 
hallmark of this style of music is to please the listener through melodic 
devices, rather than harmonic devices.

I have generally transposed these songs to be convenient to play on the 
trumpet. These transpositions may not be convenient or correct for more 
traditional Arabic instruments such as the oud or the violin.

Velvel Pasternak is a giant in Jewish sheet music. The songs that he has 
transcribed span the gamut from Chassidic to Sephardic to modern 
Israeli. His books aim to cover a breadth of different Sephardic 
traditions and include many popular Middle Eastern Sephardic songs, but 
they don't delve as deeply into Middle Eastern music as I have here.
I have a couple of his Sephardic books and have tried to avoid 
duplicating any song that he has already included in the songbooks I 
own.
As a result, there are many songs in his Sephardic songbooks 
that we do sing in Middle Eastern Sephardic synagogues, including some 
of the more popular songs, so the reader is advised to look at his 
songbooks in addition to my own.

I don't consider myself to be an expert in Arabic music, and there are 
certainly others in the shuls I frequent that have a broader knowledge of this 
genre than I do, but I hope that others will find these transcriptions helpful.

I'd like to thank my music teachers, Bill Golden Birdsong, Laurel 
Verissimo, and John Burn for teaching me music in their bands at 
Cupertino Junior High School and Homestead High School.
I'd like to thank my rabbis and 
chazzanim for teaching me the songs contained herein: Rabbi Daniel Raccah, Rabbi 
Michael Azose, Rabbi Hertzel Yitzhak, Rabbi Aharon Hamaoui, Hazzan Rabbi Shimon 
Cohen, Rabbi Aharon Raccah, and Hazzan Moshe Shemer. I'd like to thank my 
parents for encouraging me to learn music. And most of all, I'd like to thank my 
wife for supporting me and permitting me the time to accomplish all of the 
transcriptions in this book.

--Chanoch Bloom

\section*{Resources}

Full lyrics to most of the songs in this book can be found in the following seforim:

\begin{itemize}
\item \RL{אביעה רננות}.
\item \RL{שיר ושבחה} by Moshe Batzri, published by \RL{מכון הכתב} in Jerusalem.
\item \RL{שיר ושבחה חלל וזמרה}, published by the Sephardic Heritage Foundation in Brooklyn, NY.
\end{itemize}

\noindent
Recordings of most of the songs in this book can be found at:

\begin{itemize}
\item \texttt{pizmonim.org}
\item \texttt{piyut.org.il}
\item \texttt{sephardichazzanut.com}
\end{itemize}

For those interested in learning more about
Arabic music theory, I recommend \textit{Inside Arabic Music} by Johnny 
Faraj and Sami Abu Shumays, and their companion website at \texttt{www.maqamworld.com}.
Mauro Braunstein also has a great chapter on the Arabic maqamat at 
\texttt{https://offtonic.com/theory/book/7-9.html}.




\chapter{Shabbat}
\song{Ashir La'el Asher Shavat}{Ashir_Lael_Asher_Shavat.pdf}

\chapter{Turkish Melodies for Shabbat}
\song{Yoducha Rayonai}{Yoducha_Rayonai_Turkish_with_octave_up.pdf}
\song{Yom HaShabbat Ein Kamohu}{Yom_HaShabbat_Ein_Kamohu_Turkish.pdf}
\song{Ki Eshmerah Shabbat}{Ki_Eshmerah_Shabbat_Turkish.pdf}

\chapter{Bakashot for Shabbat}
\song{L'maancha v'lo Lanu}{Lmaancha_vlo_lanu.pdf}
\song{Yom Zeh Shiru La'el}{Yom_Zeh_Shiru_Lael.pdf}
\song{Yoducha Rayonai}{Yoducha_Rayonai.pdf}
\song{Yoducha Rayonai \#2}{Yoducha_Rayonai_2.pdf}
\song{Yah Ribon Alam}{Yah_Ribon_Alam.pdf}
\song{Yah Ribon Alam \#2}{Yah_Ribon_Alam_2.pdf}
\song{Yedid Nefesh}{Yedid_Nefesh.pdf}
\song{Agadelcha}{Agadelcha.pdf}

\chapter{Motzae Shabbat}
\song{Al Tira Avdi Yaakov}{Al_Tira_Avdi_Yaakov.pdf}

\chapter{Maqam Rast Family}
Maqam Rast makes use of quarter tones. When playing these songs on 
western instruments, I have found that it is usually reasonable to 
approximate the scale by sharpening the quarter tones, such that the 
scale becomes equivalent to the western major scale and Maqam Ajam.

In addition to the songs included in this section of the song book, 
here's a list of songs in this maqam family that can be found in other 
secctions of the song book:
\begin{itemize}
   \item \songref{Yah_Ribon_Alam.pdf}
\end{itemize}
\song{Eli Yah Eli}{Eli_Yah_Eli.pdf} % possibly omit, because it's unclear whether I did it in the right key
\song{Riva Riva}{Riva_Riva.pdf}
\song{El Hon}{El_Hon.pdf}
\song{Boee B'rinah}{Boee_Brinah.pdf}
\song{N'imah Li}{Nimah_Li.pdf}
\song{Rachum Bach Yagel Levavi}{Rachum_Bach_Yagel_Levavi.pdf}

\chapter{Maqam Hijaz Family}
Maqam Hijaz is considered to be equivalent to the Phrygian dominant 
mode. Maqam Hijazkar, a common variant, is known in western music theory 
as the double harmonic scale.


In addition to the songs included in this section of the song book, 
here's a list of songs in this maqam family that can be found in other 
secctions of the song book:
\begin{itemize}
   \item \songref{Im_Chacham_Libecha_Beni.pdf}
   \item \songref{Ronu_Gilu.pdf}
\end{itemize}
\song{Ana b'Chasdecha}{Ana_bChasdecha.pdf}
\song{Eleicha Ekra Yah}{Eleicha_Ekra_Yah.pdf}
\song{Ochil Yom Yom Yerushalyim}{Ochil_Yom_Yom_Yerushalyim.pdf}
\song{Ozrenei El Chai}{Ozrenei_El_Chai.pdf}
\song{El Galil}{El_Galil.pdf}

\chapter{Maqam Ajam Family}
Maqam Ajam is considered to be equivalent to the western major scale.

In addition to the songs included in this section of the song book, 
here's a list of songs in this maqam family that can be found in other 
secctions of the song book:
\begin{itemize}
   \item \songref{Yachid_Norah.pdf}
   \item \songref{Roe_Neeman_Hu.pdf}
   \item \songref{Chanun_Rachem.pdf}
   \item \songref{Yah_Hatzel_Yonah.pdf}
   \item \songref{Makhelot_Am.pdf}
\end{itemize}
\song{Refa Tziri}{Refa_Tziri.pdf}
\song{Ani Likrat}{Ani_Likrat.pdf}
\song{Ani Alayich Ayuma}{Ani_Alayich_Ayuma.pdf}
\song{Yinon Shemo}{Yinon_Shemo.pdf}
\song{Kol Et Eleicha}{Kol_Et_Eleicha.pdf}
\song{Halleluyah Tehillim 150}{Halleluyah_Tehillim_150.pdf}

\chapter{Maqam Nahawand Family}
Maqam Nahawand is considered to be equivalent to the western minor scale.

In addition to the songs included in this section of the song book, 
here's a list of songs in this maqam family that can be found in other 
secctions of the song book:
\begin{itemize}
   \item \songref{Lneri.pdf}
   \item \songref{Rachum_Atah.pdf}
   \item \songref{Yedid_Nefesh.pdf}
   \item \songref{Agadelcha.pdf}
\end{itemize}
\song{Malki Tzuri El Kabir}{Malki_Tzuri_El_Kabir.pdf}
\song{Lach Ana Orech}{Lach_Ana_Orech.pdf}
\forcefacingpages{
   \song{Im HaGolan}{Im_HaGolan.pdf}
}{
   \song{Ata El Kabir}{Ata_El_Kabir.pdf}
}
\song{Hayyim U'Madon}{Hayyim_UMadon.pdf}
\song{L'David Shir U'Mizmor}{LDavid_Shir_uMizmor.pdf}

\chapter{Maqam Kurd Family}
\song{Im Ninalu}{Im_Ninalu.pdf}
\song{Eshal Elohai}{Eshal_Elohai.pdf}
\song{Yerushalyim Ir HaBirah}{Yerushalyim_Ir_HaBirah.pdf}

\chapter{Maqam Bayat Family}
Maqam Bayat makes use of quarter tones. When playing these songs on 
western instruments, I have found that it is usually reasonable to 
approximate the scale by flattening the quarter tones, such that the 
scale becomes equivalent to the western minor scale and Maqam Nahawand.

In addition to the songs included in this section of the song book, 
here's a list of songs in this maqam family that can be found in other 
secctions of the song book:
\begin{itemize}
   \item \songref{El_Meod_Naalah.pdf}
   \item \songref{Yaalah_Yaalah.pdf}
   \item \songref{Yifat_Ayin.pdf}
   \item \songref{Nava_Yafa_Tzviyah.pdf}
   \item \songref{Ezer_Mitzarai.pdf}
   \item \songref{Daat_Umzimah.pdf}
   \item \songref{Mauzi.pdf}
\end{itemize}
\song{Yahaloma}{Yahaloma.pdf}
\song{Ana Chaper}{Ana_Chaper.pdf}
\song{Nagila Halleluyah}{Nagila_Halleluyah.pdf}
\song{Semach Beni B'Chelkecha}{Semach_Beni_BChelkecha.pdf}
\song{Chochmaah Binah Yah Eli}{Chochmah_Binah_Yah_Eli.pdf}
\song{Havivi Yah Havivi}{Havivi_Yah_Havivi.pdf}
\song{Yotzer MiYado}{Yotzer_MiYado.pdf}
\song{Eretz HaKedoshah}{Eretz_haKedoshah.pdf}

\chapter{Maqam Saba Family}
One of the pitches in Maqam Saba is a quarter tone. When playing these 
songs on western instruments, I have found that it's reasonable to 
approximate it as maqam Saba Zamzam, which sharpens the quarter tone. 
Either way, this scale doesn't really have an equivalent in western 
music theory.

In addition to the songs included in this section of the song book, 
here's a list of songs in this maqam family that can be found in other 
secctions of the song book:
\begin{itemize}
   \item \songref{Yehi_Shalom_bCheleinu_Syrian.pdf}
   \item \songref{Sukkah_vLulav_Iraqi.pdf}
   \item \songref{Ki_Eshmerah_Shabbat_Turkish.pdf}
   \item \songref{Atah_Ahuvi.pdf}
\end{itemize}

\song{Goel Yavo}{Goel_Yavo.pdf}
\song{Yedidi Roi Mikimi}{Yedidi_Roi_Mikimi.pdf}
\song{Li Yah Li Yah}{Li_Yah_Li_Yah.pdf}

\chapter{Maqam Sigah Family}
Maqam Sigah makes use of quareter tones, most notably on its tonic.
At the synangogues that I've learned melodies from, it's rare to sing 
songs Maqam Sigah when we're not borrowing their melodies for the 
Shabbat prayers.
I have transcribed a number of these songs to try and have some 
familiarity with this maqam, mainly so that I can use this maqam in the 
Shabbat prayers, however I don't yet understand it well 
enough yet to be able to explain systematically what I'm doing to 
approixmate it.

In addition to the songs included in this section of the song book, 
here's a list of songs in this maqam family that can be found in other 
secctions of the song book:
\begin{itemize}
   \item \songref{Or_Gilah.pdf}
   \item \songref{Eli_Tzur_Yishuati.pdf}
\end{itemize}
\song{Agila Agila}{Agila_Agila.pdf}
\song{Adon Yachid}{Adon_Yachid.pdf}
\song{HaYemei S'timei}{HaYemei_Stimei.pdf}
\song{Yachid El Dagul}{Yachid_El_Dagul.pdf}
\song{Nagila Halleluyah}{Nagila_Halleluyah_Sigah.pdf}

\chapter{Yamim Noraim}
\song{Lecha Eli}{Lecha_Eli.pdf}

\chapter{Shalosh Regalim}
\song{Mauzi}{Mauzi.pdf}
\song{Samachti Tehillim 122}{Samachti.pdf}
\song{B'tzeit Yisrael}{Btzeit_Yisrael.pdf}
\song{Aromimcha}{Aromimcha.pdf}

\chapter{Pesach}
\song{Yachid Norah}{Yachid_Norah.pdf}
\song{Rachum Atah}{Rachum_Atah.pdf}
\song{B'neh Li Zevul Mishkani}{Bneh_Li_Zevul_Mishkani.pdf}
\song{Emunim Irchu Shevach (Yerushalmi)}{Emunim_Irchu_Shevach_Yerushalyim.pdf}
\song{Emunim Irchu Shevach (Syrian)}{Emunim_Irchu_Shevach_Syrian.pdf}
\song{Mi Yimalel Gevurotecha}{Mi_Yimalel_Gevurotecha.pdf}
\song{El B'Yado}{El_BYado.pdf}
\song{El B'Yado \#2}{El_BYado_2.pdf}
\song{El Maleh HaNechsar}{El_Maleh_HaNechsar.pdf}
\song{Ashir laEl Ga'oh Ga'ah}{Ashir_laEl_Gaoh_Gaah.pdf}

\chapter{Lag LaOmer}
\song{V'amartem Ko Lachai}{Vamartem_Ko_Lachai.pdf}

\chapter{Shavuot}
\song{Roe' Ne'eman Hu}{Roe_Neeman_Hu.pdf}
\song{Da'at Umzimah}{Daat_Umzimah.pdf}

\chapter{Sukkot}
\song{Sukkah v'Lulav (Moroccan)}{Sukkah_vLulav_Moroccan.pdf}
\song{Sukkah v'Lulav (Yerushalmi)}{Sukkah_vLulav_Yerushalmi.pdf}
\song{Sukkah v'Lulav (Iraqi)}{Sukkah_vLulav_Iraqi.pdf}
\song{Chanun Rachem}{Chanun_Rachem.pdf}
\song{Yah Et Sukkat David Takim}{Yah_Et_Sukkat_David_Takim.pdf}
\song{Nizke l'kol Berachot (Iraqi)}{Nizke_lkol_Berachot_Bavel.pdf}
\song{Nizke l'kol Berachot}{Nizke_lkol_Berachot.pdf}

\chapter{Chanukah}
\song{Yah Hatzel Yonah}{Yah_Hatzel_Yonah.pdf}
\song{Heichalo Heichalo}{Heichalo_Heichalo.pdf}
\song{L'neri}{Lneri.pdf}
\song{Am Ne'emanai}{Am_Neemanai.pdf}
\song{Hanerot Halalu}{Hanerot_Halalu.pdf}

\chapter{Purim}
\song{Ronu Gilu}{Ronu_Gilu.pdf}
\song{Or Gilah}{Or_Gilah.pdf} % might not get included because it's in maqam sigah
\song{El Melech Ne'eman}{El_Melech_Neeman.pdf}
\song{Ezer Mitzarai}{Ezer_Mitzarai.pdf}
\song{Simeni Rosh}{Simeni_Rosh.pdf}
\song{Chish Misgabi Geulah}{Chish_Misgabi_Geulah.pdf}
\song{Eli Tzur Yishuati}{Eli_Tzur_Yishuati.pdf}

\chapter{Weddings}
\song{Yismach Hatani}{Yismach_Hatani.pdf}
\song{El Me'od Na'alah}{El_Meod_Naalah.pdf}
\song{Im Chacham Libecha Beni}{Im_Chacham_Libecha_Beni.pdf}
\song{Makhelot Am}{Makhelot_Am.pdf}
\song{Et Dodim Kalah}{Et_Dodim_Kalah.pdf}

\chapter{Brit Milah}
\song{Mah Tov Mah Na'im}{Mah_Tov_Mah_Naim.pdf}
\song{Yehi Shalom b'Cheleinu (Yerushalmi)}{Yehi_Shalom_bCheleinu_Yerushalmi.pdf}
\song{Yehi Shalom B'Cheleinu (Syrian)}{Yehi_Shalom_bCheleinu_Syrian.pdf}
\song{Atah Ahuvi}{Atah_Ahuvi.pdf}

\chapter{Zeved HaBat}
\song{Nava Yafa Tz'viyah}{Nava_Yafa_Tzviyah.pdf}
\song{Yifat Ayin}{Yifat_Ayin.pdf}
\song{Ya'alah Ya'alah}{Yaalah_Yaalah.pdf}


\end{document}
